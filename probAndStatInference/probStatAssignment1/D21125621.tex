% Options for packages loaded elsewhere
\PassOptionsToPackage{unicode}{hyperref}
\PassOptionsToPackage{hyphens}{url}
%
\documentclass[
]{article}
\usepackage{amsmath,amssymb}
\usepackage{lmodern}
\usepackage{iftex}
\ifPDFTeX
  \usepackage[T1]{fontenc}
  \usepackage[utf8]{inputenc}
  \usepackage{textcomp} % provide euro and other symbols
\else % if luatex or xetex
  \usepackage{unicode-math}
  \defaultfontfeatures{Scale=MatchLowercase}
  \defaultfontfeatures[\rmfamily]{Ligatures=TeX,Scale=1}
\fi
% Use upquote if available, for straight quotes in verbatim environments
\IfFileExists{upquote.sty}{\usepackage{upquote}}{}
\IfFileExists{microtype.sty}{% use microtype if available
  \usepackage[]{microtype}
  \UseMicrotypeSet[protrusion]{basicmath} % disable protrusion for tt fonts
}{}
\makeatletter
\@ifundefined{KOMAClassName}{% if non-KOMA class
  \IfFileExists{parskip.sty}{%
    \usepackage{parskip}
  }{% else
    \setlength{\parindent}{0pt}
    \setlength{\parskip}{6pt plus 2pt minus 1pt}}
}{% if KOMA class
  \KOMAoptions{parskip=half}}
\makeatother
\usepackage{xcolor}
\usepackage[margin=1in]{geometry}
\usepackage{color}
\usepackage{fancyvrb}
\newcommand{\VerbBar}{|}
\newcommand{\VERB}{\Verb[commandchars=\\\{\}]}
\DefineVerbatimEnvironment{Highlighting}{Verbatim}{commandchars=\\\{\}}
% Add ',fontsize=\small' for more characters per line
\usepackage{framed}
\definecolor{shadecolor}{RGB}{248,248,248}
\newenvironment{Shaded}{\begin{snugshade}}{\end{snugshade}}
\newcommand{\AlertTok}[1]{\textcolor[rgb]{0.94,0.16,0.16}{#1}}
\newcommand{\AnnotationTok}[1]{\textcolor[rgb]{0.56,0.35,0.01}{\textbf{\textit{#1}}}}
\newcommand{\AttributeTok}[1]{\textcolor[rgb]{0.77,0.63,0.00}{#1}}
\newcommand{\BaseNTok}[1]{\textcolor[rgb]{0.00,0.00,0.81}{#1}}
\newcommand{\BuiltInTok}[1]{#1}
\newcommand{\CharTok}[1]{\textcolor[rgb]{0.31,0.60,0.02}{#1}}
\newcommand{\CommentTok}[1]{\textcolor[rgb]{0.56,0.35,0.01}{\textit{#1}}}
\newcommand{\CommentVarTok}[1]{\textcolor[rgb]{0.56,0.35,0.01}{\textbf{\textit{#1}}}}
\newcommand{\ConstantTok}[1]{\textcolor[rgb]{0.00,0.00,0.00}{#1}}
\newcommand{\ControlFlowTok}[1]{\textcolor[rgb]{0.13,0.29,0.53}{\textbf{#1}}}
\newcommand{\DataTypeTok}[1]{\textcolor[rgb]{0.13,0.29,0.53}{#1}}
\newcommand{\DecValTok}[1]{\textcolor[rgb]{0.00,0.00,0.81}{#1}}
\newcommand{\DocumentationTok}[1]{\textcolor[rgb]{0.56,0.35,0.01}{\textbf{\textit{#1}}}}
\newcommand{\ErrorTok}[1]{\textcolor[rgb]{0.64,0.00,0.00}{\textbf{#1}}}
\newcommand{\ExtensionTok}[1]{#1}
\newcommand{\FloatTok}[1]{\textcolor[rgb]{0.00,0.00,0.81}{#1}}
\newcommand{\FunctionTok}[1]{\textcolor[rgb]{0.00,0.00,0.00}{#1}}
\newcommand{\ImportTok}[1]{#1}
\newcommand{\InformationTok}[1]{\textcolor[rgb]{0.56,0.35,0.01}{\textbf{\textit{#1}}}}
\newcommand{\KeywordTok}[1]{\textcolor[rgb]{0.13,0.29,0.53}{\textbf{#1}}}
\newcommand{\NormalTok}[1]{#1}
\newcommand{\OperatorTok}[1]{\textcolor[rgb]{0.81,0.36,0.00}{\textbf{#1}}}
\newcommand{\OtherTok}[1]{\textcolor[rgb]{0.56,0.35,0.01}{#1}}
\newcommand{\PreprocessorTok}[1]{\textcolor[rgb]{0.56,0.35,0.01}{\textit{#1}}}
\newcommand{\RegionMarkerTok}[1]{#1}
\newcommand{\SpecialCharTok}[1]{\textcolor[rgb]{0.00,0.00,0.00}{#1}}
\newcommand{\SpecialStringTok}[1]{\textcolor[rgb]{0.31,0.60,0.02}{#1}}
\newcommand{\StringTok}[1]{\textcolor[rgb]{0.31,0.60,0.02}{#1}}
\newcommand{\VariableTok}[1]{\textcolor[rgb]{0.00,0.00,0.00}{#1}}
\newcommand{\VerbatimStringTok}[1]{\textcolor[rgb]{0.31,0.60,0.02}{#1}}
\newcommand{\WarningTok}[1]{\textcolor[rgb]{0.56,0.35,0.01}{\textbf{\textit{#1}}}}
\usepackage{graphicx}
\makeatletter
\def\maxwidth{\ifdim\Gin@nat@width>\linewidth\linewidth\else\Gin@nat@width\fi}
\def\maxheight{\ifdim\Gin@nat@height>\textheight\textheight\else\Gin@nat@height\fi}
\makeatother
% Scale images if necessary, so that they will not overflow the page
% margins by default, and it is still possible to overwrite the defaults
% using explicit options in \includegraphics[width, height, ...]{}
\setkeys{Gin}{width=\maxwidth,height=\maxheight,keepaspectratio}
% Set default figure placement to htbp
\makeatletter
\def\fps@figure{htbp}
\makeatother
\setlength{\emergencystretch}{3em} % prevent overfull lines
\providecommand{\tightlist}{%
  \setlength{\itemsep}{0pt}\setlength{\parskip}{0pt}}
\setcounter{secnumdepth}{-\maxdimen} % remove section numbering
\usepackage{booktabs}
\usepackage{longtable}
\usepackage{array}
\usepackage{multirow}
\usepackage{wrapfig}
\usepackage{float}
\usepackage{colortbl}
\usepackage{pdflscape}
\usepackage{tabu}
\usepackage{threeparttable}
\usepackage{threeparttablex}
\usepackage[normalem]{ulem}
\usepackage{makecell}
\usepackage{xcolor}
\ifLuaTeX
  \usepackage{selnolig}  % disable illegal ligatures
\fi
\IfFileExists{bookmark.sty}{\usepackage{bookmark}}{\usepackage{hyperref}}
\IfFileExists{xurl.sty}{\usepackage{xurl}}{} % add URL line breaks if available
\urlstyle{same} % disable monospaced font for URLs
\hypersetup{
  hidelinks,
  pdfcreator={LaTeX via pandoc}}

\author{}
\date{\vspace{-2.5em}}

\begin{document}

\textbf{Student Number: D21125621}\\
\textbf{Student Name: Mauricio de Oliveira Reis}\\
\textbf{Programme Code: TU256/1}\\
\textbf{Dataset Used: Bike Sharing Dataset Data Set}

\begin{enumerate}
\def\labelenumi{\arabic{enumi}.}
\setcounter{enumi}{-1}
\tightlist
\item
  Preliminaries
\end{enumerate}

\begin{Shaded}
\begin{Highlighting}[]
\CommentTok{\# setwd as the folder where this script is in}
\NormalTok{current\_path }\OtherTok{=}\NormalTok{ rstudioapi}\SpecialCharTok{::}\FunctionTok{getActiveDocumentContext}\NormalTok{()}\SpecialCharTok{$}\NormalTok{path}
\FunctionTok{setwd}\NormalTok{(}\FunctionTok{dirname}\NormalTok{(current\_path))}
\end{Highlighting}
\end{Shaded}

\begin{Shaded}
\begin{Highlighting}[]
\CommentTok{\#Load and Install required packages }
\NormalTok{install\_packages }\OtherTok{\textless{}{-}} \ControlFlowTok{function}\NormalTok{(pkg) \{ }
  
  \CommentTok{\# Install package if it is not already}
  \ControlFlowTok{if}\NormalTok{ (}\SpecialCharTok{!}\NormalTok{(pkg }\SpecialCharTok{\%in\%} \FunctionTok{installed.packages}\NormalTok{()[, }\StringTok{"Package"}\NormalTok{]))\{ }
    
    \FunctionTok{install.packages}\NormalTok{(pkg, }\AttributeTok{repos=}\StringTok{\textquotesingle{}http://cran.us.r{-}project.org\textquotesingle{}}\NormalTok{)}
\NormalTok{  \}}
  
  \FunctionTok{library}\NormalTok{(pkg, }\AttributeTok{character.only =} \ConstantTok{TRUE}\NormalTok{)}
  
\NormalTok{\} }\CommentTok{\# end installPackages()}

\CommentTok{\#Create the list of packages we need}
\NormalTok{pkg\_list }\OtherTok{=} \FunctionTok{c}\NormalTok{(}\StringTok{"ggplot2"}\NormalTok{, }\StringTok{"dplyr"}\NormalTok{, }\StringTok{"gridExtra"}\NormalTok{, }\StringTok{"vtable"}\NormalTok{)}
\CommentTok{\#Call our function passing it the list of packages}
\FunctionTok{lapply}\NormalTok{(pkg\_list, install\_packages)}
\end{Highlighting}
\end{Shaded}

\begin{verbatim}
## 
## Attaching package: 'dplyr'
\end{verbatim}

\begin{verbatim}
## The following objects are masked from 'package:stats':
## 
##     filter, lag
\end{verbatim}

\begin{verbatim}
## The following objects are masked from 'package:base':
## 
##     intersect, setdiff, setequal, union
\end{verbatim}

\begin{verbatim}
## 
## Attaching package: 'gridExtra'
\end{verbatim}

\begin{verbatim}
## The following object is masked from 'package:dplyr':
## 
##     combine
\end{verbatim}

\begin{verbatim}
## Loading required package: kableExtra
\end{verbatim}

\begin{verbatim}
## Warning in !is.null(rmarkdown::metadata$output) && rmarkdown::metadata$output
## %in% : 'length(x) = 2 > 1' in coercion to 'logical(1)'
\end{verbatim}

\begin{verbatim}
## 
## Attaching package: 'kableExtra'
\end{verbatim}

\begin{verbatim}
## The following object is masked from 'package:dplyr':
## 
##     group_rows
\end{verbatim}

\begin{verbatim}
## [[1]]
## [1] "ggplot2"   "stats"     "graphics"  "grDevices" "utils"     "datasets" 
## [7] "methods"   "base"     
## 
## [[2]]
## [1] "dplyr"     "ggplot2"   "stats"     "graphics"  "grDevices" "utils"    
## [7] "datasets"  "methods"   "base"     
## 
## [[3]]
##  [1] "gridExtra" "dplyr"     "ggplot2"   "stats"     "graphics"  "grDevices"
##  [7] "utils"     "datasets"  "methods"   "base"     
## 
## [[4]]
##  [1] "vtable"     "kableExtra" "gridExtra"  "dplyr"      "ggplot2"   
##  [6] "stats"      "graphics"   "grDevices"  "utils"      "datasets"  
## [11] "methods"    "base"
\end{verbatim}

\begin{Shaded}
\begin{Highlighting}[]
\CommentTok{\# library(ggplot2)}
\CommentTok{\# library(dplyr)}
\CommentTok{\# require(gridExtra)}
\end{Highlighting}
\end{Shaded}

\begin{Shaded}
\begin{Highlighting}[]
\CommentTok{\#load bike sharing dataset and check its head}
\NormalTok{bike\_sharing }\OtherTok{\textless{}{-}} \FunctionTok{read.csv}\NormalTok{(}\StringTok{"bikeSharingByDay.csv"}\NormalTok{, }\AttributeTok{header=}\NormalTok{T)}
\FunctionTok{head}\NormalTok{(bike\_sharing)}
\end{Highlighting}
\end{Shaded}

\begin{verbatim}
##   instant     dteday season yr mnth holiday weekday workingday weathersit
## 1       1 2011-01-01      1  0    1       0       6          0          2
## 2       2 2011-01-02      1  0    1       0       0          0          2
## 3       3 2011-01-03      1  0    1       0       1          1          1
## 4       4 2011-01-04      1  0    1       0       2          1          1
## 5       5 2011-01-05      1  0    1       0       3          1          1
## 6       6 2011-01-06      1  0    1       0       4          1          1
##       temp    atemp      hum windspeed casual registered  cnt
## 1 0.344167 0.363625 0.805833 0.1604460    331        654  985
## 2 0.363478 0.353739 0.696087 0.2485390    131        670  801
## 3 0.196364 0.189405 0.437273 0.2483090    120       1229 1349
## 4 0.200000 0.212122 0.590435 0.1602960    108       1454 1562
## 5 0.226957 0.229270 0.436957 0.1869000     82       1518 1600
## 6 0.204348 0.233209 0.518261 0.0895652     88       1518 1606
\end{verbatim}

\begin{Shaded}
\begin{Highlighting}[]
\CommentTok{\#summary Statistics}
\FunctionTok{st}\NormalTok{(bike\_sharing)}
\end{Highlighting}
\end{Shaded}

\begin{table}

\caption{\label{tab:unnamed-chunk-4}Summary Statistics}
\centering
\begin{tabular}[t]{llllllll}
\toprule
Variable & N & Mean & Std. Dev. & Min & Pctl. 25 & Pctl. 75 & Max\\
\midrule
instant & 731 & 366 & 211.166 & 1 & 183.5 & 548.5 & 731\\
season & 731 & 2.497 & 1.111 & 1 & 2 & 3 & 4\\
yr & 731 & 0.501 & 0.5 & 0 & 0 & 1 & 1\\
mnth & 731 & 6.52 & 3.452 & 1 & 4 & 10 & 12\\
holiday & 731 & 0.029 & 0.167 & 0 & 0 & 0 & 1\\
\addlinespace
weekday & 731 & 2.997 & 2.005 & 0 & 1 & 5 & 6\\
workingday & 731 & 0.684 & 0.465 & 0 & 0 & 1 & 1\\
weathersit & 731 & 1.395 & 0.545 & 1 & 1 & 2 & 3\\
temp & 731 & 0.495 & 0.183 & 0.059 & 0.337 & 0.655 & 0.862\\
atemp & 731 & 0.474 & 0.163 & 0.079 & 0.338 & 0.609 & 0.841\\
\addlinespace
hum & 731 & 0.628 & 0.142 & 0 & 0.52 & 0.73 & 0.973\\
windspeed & 731 & 0.19 & 0.077 & 0.022 & 0.135 & 0.233 & 0.507\\
casual & 731 & 848.176 & 686.622 & 2 & 315.5 & 1096 & 3410\\
registered & 731 & 3656.172 & 1560.256 & 20 & 2497 & 4776.5 & 6946\\
cnt & 731 & 4504.349 & 1937.211 & 22 & 3152 & 5956 & 8714\\
\bottomrule
\end{tabular}
\end{table}

\hypertarget{does-the-temperature-actual-impact-the-total-number-of-bikes-hired-per-day}{%
\subsection{1. Does the temperature (actual) impact the total number of
bikes hired per
day?}\label{does-the-temperature-actual-impact-the-total-number-of-bikes-hired-per-day}}

\begin{itemize}
\tightlist
\item
  Null hypothesis \((H_0)\): Temperature (actual) doesn't impact the
  total number of bikes hired per day\\
\item
  Alternative hypothesis \(H_\alpha\): Temperature (actual) impacts the
  total number of bikes hired per day\\
\item
  Significance level \(\alpha\): 0.05
\end{itemize}

\begin{Shaded}
\begin{Highlighting}[]
\CommentTok{\#Check if there are NAs on the dataset}
\NormalTok{bike\_sharing[}\SpecialCharTok{!}\FunctionTok{complete.cases}\NormalTok{(bike\_sharing),]}
\end{Highlighting}
\end{Shaded}

\begin{verbatim}
##  [1] instant    dteday     season     yr         mnth       holiday   
##  [7] weekday    workingday weathersit temp       atemp      hum       
## [13] windspeed  casual     registered cnt       
## <0 rows> (or 0-length row.names)
\end{verbatim}

Let's fist check the Daily temperature distribution and see how normal
looking it's:

\begin{Shaded}
\begin{Highlighting}[]
\NormalTok{plot\_with\_mean\_and\_median }\OtherTok{\textless{}{-}} \ControlFlowTok{function}\NormalTok{(plot, df, attribute) \{}
  \FunctionTok{return}\NormalTok{(plot }\SpecialCharTok{+} \FunctionTok{geom\_vline}\NormalTok{(}\FunctionTok{aes}\NormalTok{(}\AttributeTok{xintercept =} \FunctionTok{mean}\NormalTok{(df[[attribute]])),}\AttributeTok{col=}\StringTok{\textquotesingle{}red\textquotesingle{}}\NormalTok{, }\AttributeTok{size=}\DecValTok{1}\NormalTok{) }\SpecialCharTok{+}
  \FunctionTok{geom\_vline}\NormalTok{(}\FunctionTok{aes}\NormalTok{(}\AttributeTok{xintercept =} \FunctionTok{median}\NormalTok{(df[[attribute]])),}\AttributeTok{col=}\StringTok{\textquotesingle{}blue\textquotesingle{}}\NormalTok{, }\AttributeTok{size=}\DecValTok{1}\NormalTok{))}
\NormalTok{\}}

\CommentTok{\#Check distribution of temperature }
\NormalTok{shapTestPValue }\OtherTok{\textless{}{-}} \FunctionTok{shapiro.test}\NormalTok{(bike\_sharing}\SpecialCharTok{$}\NormalTok{temp)}\SpecialCharTok{$}\NormalTok{p.value}
\NormalTok{p1}\OtherTok{\textless{}{-}}\FunctionTok{qplot}\NormalTok{(bike\_sharing}\SpecialCharTok{$}\NormalTok{temp, }\AttributeTok{main =} \StringTok{"Daily temperature distribution"}\NormalTok{, }\AttributeTok{xlab =} \StringTok{"Temperature (Normalized)"}\NormalTok{, }\AttributeTok{binwidth=}\FloatTok{0.05}\NormalTok{) }

\FunctionTok{plot\_with\_mean\_and\_median}\NormalTok{(p1, bike\_sharing,}\StringTok{"temp"}\NormalTok{)}
\end{Highlighting}
\end{Shaded}

\includegraphics{D21125621_files/figure-latex/unnamed-chunk-6-1.pdf}
This distribution doesn't look that much like a normal one, even though
the mean(red) and the median(blue) are quite close. In fact, the
Shapiro-Wilk test gave a p-value of \texttt{shapTestPValue}, way below
the 0.05 that would indicate a normal distribution.

Let's also check how our users are distributed:

\begin{Shaded}
\begin{Highlighting}[]
\CommentTok{\#Check the amount of users for each type}
\NormalTok{casualAndRegisteredDf }\OtherTok{\textless{}{-}}\NormalTok{ bike\_sharing }\SpecialCharTok{\%\textgreater{}\%} \FunctionTok{select}\NormalTok{(casual, registered)}
\NormalTok{sumUserTypeDf}\OtherTok{\textless{}{-}} \FunctionTok{data.frame}\NormalTok{(}\AttributeTok{value=}\FunctionTok{apply}\NormalTok{(casualAndRegisteredDf,}\DecValTok{2}\NormalTok{,sum))}
\NormalTok{sumUserTypeDf}\SpecialCharTok{$}\NormalTok{key}\OtherTok{=}\FunctionTok{rownames}\NormalTok{(sumUserTypeDf)}
\NormalTok{p }\OtherTok{\textless{}{-}}\FunctionTok{ggplot}\NormalTok{(}\AttributeTok{data=}\NormalTok{sumUserTypeDf, }\FunctionTok{aes}\NormalTok{(}\AttributeTok{x=}\NormalTok{key, }\AttributeTok{y=}\NormalTok{value, }\AttributeTok{fill=}\NormalTok{key)) }\SpecialCharTok{+}
\FunctionTok{geom\_bar}\NormalTok{(}\AttributeTok{colour=}\StringTok{"black"}\NormalTok{, }\AttributeTok{stat=}\StringTok{"identity"}\NormalTok{, }\AttributeTok{show.legend=}\ConstantTok{FALSE}\NormalTok{) }
\NormalTok{p }\SpecialCharTok{+} \FunctionTok{labs}\NormalTok{(}\AttributeTok{title =} \StringTok{"Quantity of casual and registered users"}\NormalTok{, }\AttributeTok{x=}\StringTok{"Type of user"}\NormalTok{, }\AttributeTok{y=}\StringTok{"Quantity"}\NormalTok{)}
\end{Highlighting}
\end{Shaded}

\includegraphics{D21125621_files/figure-latex/unnamed-chunk-7-1.pdf}

\begin{Shaded}
\begin{Highlighting}[]
\NormalTok{totalCasual }\OtherTok{\textless{}{-}} \FunctionTok{sum}\NormalTok{(bike\_sharing}\SpecialCharTok{$}\NormalTok{casual)}
\NormalTok{totalRegistered }\OtherTok{\textless{}{-}} \FunctionTok{sum}\NormalTok{(bike\_sharing}\SpecialCharTok{$}\NormalTok{registered)}
\NormalTok{total }\OtherTok{\textless{}{-}}\NormalTok{ totalCasual }\SpecialCharTok{+}\NormalTok{ totalRegistered}
\NormalTok{percentCasual }\OtherTok{\textless{}{-}} \FunctionTok{round}\NormalTok{(totalCasual }\SpecialCharTok{*} \DecValTok{100} \SpecialCharTok{/}\NormalTok{ total, }\DecValTok{2}\NormalTok{ )}
\NormalTok{percentRegistered }\OtherTok{\textless{}{-}} \FunctionTok{round}\NormalTok{(totalRegistered }\SpecialCharTok{*} \DecValTok{100} \SpecialCharTok{/}\NormalTok{ total, }\DecValTok{2}\NormalTok{)}
\end{Highlighting}
\end{Shaded}

\texttt{percentCasual}\% of users are casual and
\texttt{percentRegistered}\% are registered. As we can see, the vast
majority of our users are registered users.

How each type of user distribution looks like? (TODO shapiro-wild test
those and read about the importance of normality to stablish
correlation)

\begin{Shaded}
\begin{Highlighting}[]
\CommentTok{\#create a new column with the sum of casual and registered uses}
\NormalTok{bike\_sharing}\SpecialCharTok{$}\NormalTok{casualAndRegistered }\OtherTok{\textless{}{-}}\NormalTok{ bike\_sharing}\SpecialCharTok{$}\NormalTok{casual }\SpecialCharTok{+}\NormalTok{ bike\_sharing}\SpecialCharTok{$}\NormalTok{registered}

\CommentTok{\#Check distribution of number of bikes hired per day }
\NormalTok{p1 }\OtherTok{\textless{}{-}} \FunctionTok{plot\_with\_mean\_and\_median}\NormalTok{(}
  \FunctionTok{qplot}\NormalTok{(bike\_sharing}\SpecialCharTok{$}\NormalTok{casual, }\AttributeTok{main =} \StringTok{"Daily casual users distribution"}\NormalTok{, }\AttributeTok{xlab =} \StringTok{"Casual users"}\NormalTok{),}
\NormalTok{  bike\_sharing,}
  \StringTok{"casual"}\NormalTok{)}
\NormalTok{p2 }\OtherTok{\textless{}{-}} \FunctionTok{plot\_with\_mean\_and\_median}\NormalTok{(}
  \FunctionTok{qplot}\NormalTok{(bike\_sharing}\SpecialCharTok{$}\NormalTok{registered, }\AttributeTok{main =} \StringTok{"Daily registered users distribution"}\NormalTok{, }\AttributeTok{xlab =} \StringTok{"Registered users"}\NormalTok{),}
\NormalTok{  bike\_sharing,}
  \StringTok{"registered"}\NormalTok{)}
\NormalTok{p3 }\OtherTok{\textless{}{-}} \FunctionTok{plot\_with\_mean\_and\_median}\NormalTok{(}
  \FunctionTok{qplot}\NormalTok{(bike\_sharing}\SpecialCharTok{$}\NormalTok{casualAndRegistered, }\AttributeTok{main =} \StringTok{"Daily total users distribution"}\NormalTok{, }\AttributeTok{xlab =} \StringTok{"Users"}\NormalTok{),}
\NormalTok{  bike\_sharing,}
  \StringTok{"casualAndRegistered"}\NormalTok{)}
\FunctionTok{grid.arrange}\NormalTok{(p1, p2, p3, }\AttributeTok{nrow =} \DecValTok{3}\NormalTok{)}
\end{Highlighting}
\end{Shaded}

\begin{verbatim}
## `stat_bin()` using `bins = 30`. Pick better value with `binwidth`.
## `stat_bin()` using `bins = 30`. Pick better value with `binwidth`.
## `stat_bin()` using `bins = 30`. Pick better value with `binwidth`.
\end{verbatim}

\includegraphics{D21125621_files/figure-latex/unnamed-chunk-9-1.pdf}
Daily casual users is a right skewed distribution whereas registered
users is a little bit more normal looking. The combination of the two
gives a more spread distribution.

\begin{Shaded}
\begin{Highlighting}[]
\NormalTok{p1 }\OtherTok{\textless{}{-}} \FunctionTok{ggplot}\NormalTok{(bike\_sharing, }\FunctionTok{aes}\NormalTok{(}\AttributeTok{x=}\NormalTok{temp, }\AttributeTok{y=}\NormalTok{casual)) }\SpecialCharTok{+} \FunctionTok{geom\_point}\NormalTok{() }\SpecialCharTok{+} \FunctionTok{geom\_smooth}\NormalTok{(}\AttributeTok{method=}\NormalTok{lm) }\SpecialCharTok{+} 
  \FunctionTok{labs}\NormalTok{(}\AttributeTok{title =} \StringTok{"Casual users vs. Temperature"}\NormalTok{, }\AttributeTok{x=}\StringTok{"Temperature"}\NormalTok{, }\AttributeTok{y=}\StringTok{"Quantity of Users"}\NormalTok{)}

\NormalTok{p2 }\OtherTok{\textless{}{-}} \FunctionTok{ggplot}\NormalTok{(bike\_sharing, }\FunctionTok{aes}\NormalTok{(}\AttributeTok{x=}\NormalTok{temp, }\AttributeTok{y=}\NormalTok{registered)) }\SpecialCharTok{+} \FunctionTok{geom\_point}\NormalTok{() }\SpecialCharTok{+} \FunctionTok{geom\_smooth}\NormalTok{(}\AttributeTok{method=}\NormalTok{lm) }\SpecialCharTok{+} 
  \FunctionTok{labs}\NormalTok{(}\AttributeTok{title =} \StringTok{"Registered users vs. Temperature"}\NormalTok{, }\AttributeTok{x=}\StringTok{"Temperature"}\NormalTok{, }\AttributeTok{y=}\StringTok{"Quantity of Users"}\NormalTok{)}


\NormalTok{p3 }\OtherTok{\textless{}{-}} \FunctionTok{ggplot}\NormalTok{(bike\_sharing, }\FunctionTok{aes}\NormalTok{(}\AttributeTok{x=}\NormalTok{temp, }\AttributeTok{y=}\NormalTok{casualAndRegistered)) }\SpecialCharTok{+} \FunctionTok{geom\_point}\NormalTok{() }\SpecialCharTok{+} \FunctionTok{geom\_smooth}\NormalTok{(}\AttributeTok{method=}\NormalTok{lm) }\SpecialCharTok{+} 
  \FunctionTok{labs}\NormalTok{(}\AttributeTok{title =} \StringTok{"All users vs. Temperature"}\NormalTok{, }\AttributeTok{x=}\StringTok{"Temperature"}\NormalTok{, }\AttributeTok{y=}\StringTok{"Quantity of Users"}\NormalTok{)}

\FunctionTok{grid.arrange}\NormalTok{(p1, p2, p3, }\AttributeTok{nrow =} \DecValTok{2}\NormalTok{)}
\end{Highlighting}
\end{Shaded}

\begin{verbatim}
## `geom_smooth()` using formula 'y ~ x'
## `geom_smooth()` using formula 'y ~ x'
## `geom_smooth()` using formula 'y ~ x'
\end{verbatim}

\includegraphics{D21125621_files/figure-latex/unnamed-chunk-10-1.pdf}
calculate as skewness and kutosis. If outside of + or -2, we convert the
data to standardized score, and then we see how many rows fall outside
3.29, if less than 5\% fall outside, we can treat as normal those two
variables. Find in the slides how to report these findings. If both
variables follow normal: pick pearson - A person correlation test was
conducted\ldots{} If one or both are not normal, we can use spearman
test - A spearman correlation was found ..

if p 0.1 or 0.3 is weak\ldots{} if 0.5 and\ldots{} - Humidity 0 is
missing data

\begin{itemize}
\item
  working day, categorical varibla with two values, so this is a
  \ldots{} test.

  \begin{itemize}
  \tightlist
  \item
    use describeby for each of the groups of working day
  \end{itemize}
\item
  If normal distribution, we use a t-test. If not normal, we use a
  manwitney??Test
\item
  Use anova test if number of bikes is normally distributed for the per
  weekday
\item
  weather situation related to season?

  \begin{itemize}
  \tightlist
  \item
    two nominal variables, not ordinal
  \item
    chi-squared test
  \end{itemize}
\end{itemize}

\end{document}
